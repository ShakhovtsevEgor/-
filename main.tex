%------------------Settings-------------------------

\documentclass[12pt]{article}
\usepackage[utf8]{inputenc}
\usepackage[russian]{babel}
\usepackage{amsmath,amssymb}
\usepackage{graphics}
\usepackage{pbox}
\usepackage[x11names]{xcolor}
\definecolor{brightmaroon}{rgb}{0.76, 0.13, 0.28}
\definecolor{royalazure}{rgb}{0.0, 0.22, 0.66}
\usepackage[colorlinks=true,linkcolor=royalazure]{hyperref}
\usepackage{tikz, tkz-fct, pgfplots}
\usetikzlibrary{arrows}
\usepackage{geometry}
\geometry{
a4paper,
total={170mm,257mm},
left=20mm,
top=20mm
}
\usepackage[labelsep=period]{caption}

% —-------------— Commands —-------------—

\newcommand{\eps}{\varepsilon}
\newcommand\tline[2]{$\underset{\text{#1}}{\text{\underline{\hspace{#2}}}}$}

% —-------------— Set graphics path —-------------—
\graphicspath{{img/}}
\begin{document}
\pagestyle{empty}

% —--------------------Title----------------------------------
\centerline{\large Министерство науки и высшего образования}
\centerline{\large Федеральное государственное бюджетное образовательное}
\centerline{\large учреждение высшего образования}
\centerline{\large ``Московский государственный технический университет}
\centerline{\large имени Н.Э. Баумана}
\centerline{\large (национальный исследовательский университет)''}
\centerline{\large (МГТУ им. Н.Э. Баумана)}
\hrule
\vspace{0.5cm}
\begin{figure}[h]
\center
\includegraphics[height=0.35\linewidth]{bmstu-logo-small.png}
\end{figure}
\begin{center}
\large
\begin{tabular}{c}
Факультет ``Фундаментальные науки'' \\
Кафедра ``Высшая математика''
\end{tabular}
\end{center}
\vspace{0.5cm}
\begin{center}
\LARGE \bf
\begin{tabular}{c}
\textsc{Отчёт} \\
по учебной практике \\
за 1 семестр 2020---2021 гг.
\end{tabular}
\end{center}
\vspace{0.5cm}
\begin{center}
\large
\begin{tabular}{p{5.3cm}ll}
\pbox{5.45cm}{
Руководитель практики,\\
ст. преп. кафедры ФН1} & \tline{\it(подпись)}{5cm} & Кравченко О.В. \\[0.5cm]
студент группы ФН1--11 & \tline{\it(подпись)}{5cm} & Шаховцев Е.А.
\end{tabular}
\end{center}
\vfill
\begin{center}
\large
\begin{tabular}{c}
Москва, \\
2020 г.
\end{tabular}
\end{center}
\newpage
\newpage
\tableofcontents
%------------------Table of contents----------------------
\newpage
\section{Цели и задачи практики}
\subsection{Цели}
-— развитие компетенций, способствующих успешному освоению материала бакалавриата и необходимых в будущей профессиональной деятельности.
\subsection{Задачи}
\begin{enumerate}
\item Знакомство с программными средствами, необходимыми в будущей профессиональной деятельности.
\item Развитие умения поиска необходимой информации в специальной литературе и других источниках.
\item Развитие навыков составления отчётов и презентации результатов.
\end{enumerate}
\subsection{Индивидуальное задание}
\begin{enumerate}
\item Изучить способы отображения математической информации в системе вёртски \LaTeX.
\item Изучить возможности системы контроля версий \textsf{Git}.
\item Научиться верстать математические тексты, содержащие формулы и графики в системе \LaTeX.
Для этого, выполнить установку свободно распространяемого дистрибутива \textsf{TeXLive} и оболочки \textsf{TeXStudio}.
\item Оформить в системе \LaTeX типовые расчёты по курсе математического анализа согласно своему варианту.
\item Создать аккаунт на онлайн ресурсе \textsf{GitHub} и загрузить исходные \textsf{tex}--файлы
и результат компиляции в формате \textsf{pdf}.
\end{enumerate}
%---------------------------------------------------------------
\newpage
\section{Отчёт}
Актуальность темы продиктована необходимостью владеть системой вёрстки \LaTeX и средой вёрстки \textsf{TeXStudio} для
отображения текста, формул и графиков. Полученные в ходе практики навыки могут быть применены при написании
курсовых проектов и дипломной работы, а также в дальнейшей профессиональной деятельности.
Ситема вёрстки \LaTeX содержит большое количество инструментов (пакетов), упрощающих отображение информации в различных
сферах инженерной и научной деятельности.
 
%-----------------------------------------------------------------
\newpage
\section{Индивидуальное задание}
%\subsection{Элементарные функции и их графики.}
%\input{src/part1.tex}
%==============================================================================
\subsection{Пределы и непрерывность.}
%--------------------------— Problem 1----------------------------------
\subsubsection*{\center Задача № 1.}

%=====================================================================
\textbf{Условие:}
Дана последовательность $a_{n}=\dfrac{23-4n}{2-n}$ и число $c=4$. Доказать,что $\lim\limits_{x\rightarrow\infty} a_{n}=c$, а именно для кажого $\eps > 0$ найти наименьшее натуральное число $N = N(\eps)$ такое, что $|a_{n}-c|<\eps$. Заполнить таблицу:
\begin{table}[h]
\centering
\begin{tabular}{|c|c|c|c|}
\hline
$\eps$ & $0,1$ & $0,01$ & $0,001$ \\
\hline
$N(\eps)$ & & & \\
\hline
\end{tabular}
\end{table}
\textbf{Решение:}
$$a_{n}=\dfrac{23-4n}{2-n}; \; c = 4.$$
Найдём предел $a_{n}$:
$$\lim\limits_{x\rightarrow\infty} a_{n}= 4 = c.$$
Рассмотрим $|a_{n}-c|<\eps$:
$$\biggl |\dfrac{4n-23}{n-2} - 4 \biggr |<\eps,$$
$$\biggl |\dfrac{(4n-23)-(4n-8)}{n-2} \biggr | <\eps,$$
$$\dfrac{15}{n-2} < \eps,$$
$$ n> \dfrac{15+2\eps}{\eps}, $$
При $\eps = 0,1$ получим:
$$ n > \dfrac{15+2*0,1}{0,1} \Leftrightarrow n > 152.$$
При $\eps = 0,01$ получим:
$$ n > \dfrac{15+2*0,01}{2*0,01} \Leftrightarrow n > 1502.$$
При $\eps = 0,001$ получим:
$$ n > \dfrac{15+2*0,001}{2*0,001} \Leftrightarrow n > 15002.$$
Заполним таблицу:
\begin{table}[h]
\centering
\begin{tabular}{|c|c|c|c|}
\hline
$\eps$ & $0,1$ & $0,01$ & $0,001$ \\
\hline
$N(\eps)$ & $152$ & $1502$ & $15002$ \\
\hline
\end{tabular}
\end{table}
\newpage
% —------------------------— Problem 2----------------------------------

\begin{center}
\textbf{Задача № 2.}
\end{center}
\textbf{Условие:}
Вычислить пределы функций
\begin{table}[h]
\centering
\begin{tabular}{|c|c|}
\hline
a & $\lim\limits_{x\rightarrow 1} \dfrac{x^3-3x+2}{x^3-4x+3}$ \\
\hline
б & $\lim\limits_{x\rightarrow \infty} \dfrac{x^2+4x^3+5}{\sqrt{x+\sqrt{x^2+16x^{12}}}}$ \\
\hline
в & $\lim\limits_{x\rightarrow 7} \dfrac{\sqrt{x+2}-\sqrt{2x-5}}{\sqrt[3]{x+1}-2}$\\
\hline
г & $\lim\limits_{x\rightarrow {2\pi}}(\cos{x}+\sin^2{x})^{\frac{\cot{2x}}{\sin{3x}}}$\\
\hline
д & $\lim\limits_{x\rightarrow 0} (\dfrac{\ln{(1+2x)}}{e^{3x}-e^{2x}})^{\frac{x-1}{x-2}}$ \\
\hline
е & $\lim\limits_{x\rightarrow \pi} \dfrac{\sin^2{x}-\tan^2{x}}{(x-\pi)^4}$\\
\hline
\end{tabular}
\end{table}

\textbf{Решение:}\\
а)
$$\lim\limits_{x\rightarrow 1} \dfrac{x^3-3x+2}{x^3-4x+3}.$$
Получаем неопределённость: $$\biggl[\dfrac{0}{0}\biggr].$$
Разложим на множители:
$$\lim\limits_{x\rightarrow 1} \dfrac{(x-1)(x^2+x-2)}{(x-1)(x^2+x-3)} .$$
Сокращаем одинаковые множители:
$$\lim\limits_{x\rightarrow 1} \dfrac{x^2+x-2}{x^2+x-3}=\dfrac{0}{-1}=0.$$

б)
$$\lim\limits_{x\rightarrow\infty} \dfrac{x^2+4x^3+5}{\sqrt{x+\sqrt{x^2+16x^{12}}}}.$$
Получаем неопределённость: $$\biggl[\dfrac{\infty}{\infty}\biggr].$$
Делим на $2x^3$ числитель и знаменатель:
$$2*\lim\limits_{x\rightarrow\infty} \dfrac{1+\dfrac{1}{4x}+\dfrac{5}{4x^3}}{\sqrt{\dfrac{1}{2x^2}+\sqrt{1+\dfrac{1}{4x^2}}}} = 2.$$

\text{\bf(в):}
$$
\begin{array}{1}
\\
\bigskip
\lim\limits_{x\rightarrow7} \dfrac{\sqrt{x+2}-\sqrt{2x-5}}{\sqrt[3]{x+1}-2} = \left[\dfrac{0}{0} \right]=
\\
\bigskip
=\lim\limits_{x\rightarrow7}\dfrac{(\sqrt{x+2}-\sqrt{2x-5})(\sqrt{x+2}+\sqrt{2x-5})(\sqrt[3]{(x+1)^2}+2\sqrt[3]{x+1}+4)}{(\sqrt[3]{x+1}-2)(\sqrt[3]{x+1}+2\sqrt[3]{x+1}+4)(\sqrt{x+2}+\sqrt{2x-5})}=
\\
\bigskip
=-\lim\limits_{x\rightarrow7} \dfrac{\sqrt[3]{(x+1)^2}+2\sqrt[3]{x+1}+4}{\sqrt{x+2}+\sqrt{2x+5}}=\dfrac{\sqrt[3]{64}+2\sqrt[3]{8}+4}{\sqrt{9}+\sqrt{9}}=
\\
=\dfrac{12}{6}=2
\end{array}
$$
\\
\text{\bf(г):}
$$
\begin{array}{l}
\lim\limits_{x\rightarrow2\pi} \biggl((1+(\sin^2{x}-2\sin^2{\dfrac{x}{2}})^{\dfrac{1}{\sin^2{x}-2\sin^2{\dfrac{x}{2}}}}\biggr)^{\frac{\cos{2x}(\sin^2{x}-2\sin^2{\dfrac{x}{2}})}{\sin{2x}-\sin{3x}}}=
\\
\bigskip
=e^\lim\limits_{x\rightarrow0} \dfrac{\cos{2x}(\sin^2{x}-2\sin^2{\dfrac{x}{2})}}{\sin{2x}*\sin{3x}}} =e^{\lim\limits_{x\rightarrow0} \dfrac{x^2-2\dfrac{x^2}{4}}{2x*3x}=
\\
\bigskip
=e^{\lim\limits_{x\rightarrow0}\dfrac{x^2}{12x^2}}=e^\dfrac{1}{12}
\end{array}
$$
\\
\text{\bf(д):}
$$
\begin{array}{l}
\lim\limits_{x\rightarrow0} \dfrac{\ln{1+2x}}{e^x-1}=\limlimits_{x\rightarrow0} \dfrac{\ln{1+2x}}{e^{2x}-1}*\lim\limits_{x_\rightarrow0}\dfrac{e^{2x}-1}{e^x-1}=\lim\limits_{x\rightarrow0} \dfrac{(e^x-1)(e^x+1)}{e^x-1}=\lim\limits_{x\rightarrow0} (e^x+1)=2
\end{array}
$$
\text{\bf(е):}
$$
\begin{array}{l}
\lim\limits_{x\rightarrow\pi} \dfrac{\sin^2{x}-\tan^2{x}}{(x-\pi)^4}=\left| y=x-\pi, y \to 0\right|=\lim\limits_{y\rightarrow0} \dfrac{\sin^2{y}(1-\dfrac{1}{\cos^2{y}})}{y^4}=\lim\limits_{x\rightarrow0} \dfrac{-\sin^2{y}}{y^2}=-1
\end{array}
$$
% —------------------------— Problem 3----------------------------------
\subsubsection*{\center Задача № 3.}
{\bf Условие.~}\\
\text{\bf(а):} Показать, что данные функции
$f(x)$ и $g(x)$ являются бесконечно малыми или бесконечно большими
при указанном стремлении аргумента. \\
\text{\bf(б):} Для каждой функции $f(x)$ и $g(x)$ записать главную часть
(эквивалентную ей функцию) вида $C(x-x_0)^{\alpha}$ при $x\rightarrow x_0$ или $Cx^{\alpha}$
при $x\rightarrow\infty$, указать их порядки малости (роста). \\
\text{\bf(в):} Сравнить функции $f(x)$ и $g(x)$ при указанном стремлении.
\begin{center}
\begin{tabular}{|c|c|c|}
\hline
№ варианта & функции $f(x)$ и $g(x)$ & стремление \\[6pt]
\hline
26 & $f(x) = 2^x-1,~g(x)=\ln{(1+\sqrt{x+\sin{x}})} & $x\rightarrow0 \\
\hline
\end{tabular}
\bigskip
\\
{\bf Решение.~}\\
\end{center}
\medskip
\text{\bf(а):}~Покажем, что $f(x)$ и $g(x)$ бесконечно малые функции,
$$
\begin{array}{l}
\lim\limits_{x\rightarrow0} f(x)=2^0-1=0
\\
\lim\limits_{x\rightarrow0} g(x)=\ln{(1+\sqrt{0+0})}=\ln{1}=0
\end{array}
$$
\text{\bf(б):}
Рассмотрим предел
\\
\bigskip
\lim\limits_{x\rightarrow0} $$\dfrac{2^x-1}{x\ln{2}}=\lim\limits_{x\rightarrow0}\dfrac{e^{x\ln{2}}-1}{x\ln{2}}=1
\\
\bigskip

f(x)\sim {\ln{2}x}~x\rightarrow0.
\\
\bigskip

\lim\limits_{x\rightarrow0}\dfrac{\ln{1+\sqrt{x+\sin{x}}}}{\sqrt{2x}} =\lim\limits_{x\rightarrow0}\dfrac{\ln{1+\sqrt{2x}}}{\sqrt{2x}}=1
\\
\bigskip

g(x)\sim\sqrt{2x} \text{при}~x\rightarrow0.
\\
\bigskip

\text{(в) }\lim\limits_{x\rightarrow0}\dfrac{f(x)}{g(x)}.

\lim\limits_{x\rightarrow0}\dfrac{f(x)}{g(x)} = \lim\limits_{x\rightarrow0}\dfrac{2^x-1}{\ln{1+\sqrt{x=\sin{x}}}}=\lim\limits_{x\rightarrow0} \dfrac{\ln2*x}{\sqrt{2x}}=\lim\limits_{x\rightarrow0} \dfrac{\ln{2}}{\sqrt{2}}*x^{\dfrac{1}{2}}=0
$$

%=================================================================================================================================
%\subsection{Приложения дифференциального
 

%\input{src/part3.tex}
\newpage
\addcontentsline{toc}{section}{Список литературы}
\begin{thebibliography}{99}
\bibitem{book01} Львовский С.М. Набор и вёрстка в системе \LaTeX, 2003 c.
\bibitem{book02} Е.М. Балдин Компьютерная типография \LaTeX.
\end{thebibliography}
\end{document}